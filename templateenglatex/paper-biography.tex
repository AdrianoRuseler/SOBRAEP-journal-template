
\section*{BIOGRAPHIES}

\textbf{\underline{Adriano Ruseler}}
was born in Presidente Getúlio, Santa Catarina, Brazil, in 1982. He received the B.S. and M.S. degrees in Electrical Engineering from 
the Federal University of Santa Catarina, Florianópolis, Brazil,  in 2008 and 2011 respectively.

He is currently a Ph.D. student at the Power Electronics Institute, Federal University of Santa Catarina, Brazil. His interests include power conversion, AC machines modeling, power converter modeling, and renewable energy sources.

M. Eng. Ruseler is a student member of the Brazilian Power Electronic Society (SOBRAEP), IEEE Power Electronics Society (PELS) and IEEE Industrial Electronics Society (IES). 

\vspace{12pt}

\textbf{\underline{Telles Brunelli Lazzarin}}
was born in Criciúma, Santa Catarina State, Brazil, in 1979. He received the B.Sc., M.Sc. and Ph.D. degrees in Electrical Engineering from the Federal University of Santa Catarina (UFSC), Florian\'opolis, Brazil, in 2004, 2006 and 2010, respectively.

He is currently an Adjunct Professor at the Federal University of Santa Catarina (UFSC), Florianópolis, Brazil.  His interests include inverters, parallel operation of inverters, UPS, high-voltage dc-dc converters, direct ac-ac power converters, switched-capacitor converters and hybrid switched-capacitor converters.

Prof. Dr. Lazzarin is a member of the Brazilian Power Electronic Society (SOBRAEP), IEEE Power Electronics Society (PELS) and IEEE Industrial
Electronics Society (IES). 

\vspace{12pt}

\textbf{\underline{Ivo Barbi}} was born in Gaspar, Santa Catarina, Brazil, in 1949. He received a B.S. and a M.S. degree in Electrical Engineering from the Federal University of Santa Catarina, Florianópolis, Brazil, in 1973 and 1976, respectively, and the Dr. Ing. degree from the Institut National Polytechnique de Toulouse, France, in 1979.

He founded the Brazilian Power Electronics Society and the Power Electronics Institute of the Federal University of Santa Catarina.
He is currently a visiting professor of the Department of Automation and Systems (DAS), Federal University of Santa Catarina.

Prof. Dr. Barbi is a member of SOBRAEP, IEEE Power Electronics Society and IEEE Industrial Electronics Society.




